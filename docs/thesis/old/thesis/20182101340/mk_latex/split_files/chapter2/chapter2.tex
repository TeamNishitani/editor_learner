\chapter{基本的事項}\label{ux57faux672cux7684ux4e8bux9805}

    \section{Emacs}\label{emacs}

    本研究において使用するeditorはEmacsである.ツールはプログラマ自身の手の延長である.これは他のどのようなソフトウェアツールよりもEditorに対して当てはまる.テキストはプログラミングにおける最も基本的な生素材なので,できる限り簡単に操作できる必要があります.

そこで西谷研究室で勧められているEmacsの機能については以下の通りである,

\begin{enumerate}
\def\labelenumi{\arabic{enumi}.}
\tightlist
\item
  設定可能である. フォント,色,ウィンドウサイズ,キーバインドを含めた全ての外見が好みに応じて設定できるようになっていること.通常の操作がキーストロークだけで行えると,手をキーボードから離す必要がなくなり,結果的にマウスやメニュー駆動型のコマンドよりも効率的に操作できるようになります
\item
  拡張性がある. 新しいプログラミング言語が出てきただけで,使い物にならなくなるようなエディタではなく,どんな新しい言語やテキスト形式が出てきたとしても,その言語の意味合いを「教え込む」ことが可能です
\item
  プログラム可能であること. 込み入った複数の手順を実行できるよう,Editorはプログラム可能であることが必須である.
\end{enumerate}

これらの機能は本来エディタが持つべき基本的な機能である.これらに加えてEmacsは,

\begin{enumerate}
\def\labelenumi{\arabic{enumi}.}
\tightlist
\item
  構文のハイライト Rubyの構文にハイライトを入れたい場合はファイル名の後に.rbと入れることでRubyモードに切り替わり構文にハイライトを入れることが可能になる.
\item
  自動インデント. テキストを編集する際,改行時に自動的にスペースやタブなどを入力しインデント調整を行ってくれる.
\end{enumerate}

などのプログラミング言語に特化した特徴を備えています.強力なeditorを習熟することは生産性を高めることに他ならない.カーソルの移動にしても,1回のキー入力で単語単位,行単位,ブロック単位,関数単位でカーソルを移動させることができれば,一文字ずつ,あるいは一行ずつ繰り返してキー入力を行う場合とは効率が大きく変わってきます.Emacsはこれらの全ての機能を孕んでいてeditorとして非常に優秀である.よって本研究はEmacsをベースとして研究を進める.

    \section{Ruby}\label{ruby}

Rubyの基本的な説明は以下の通り,Rubyはまつもとゆきひろにより開発されたオブジェクト指向スクリプト言語であり,スクリプト言語が用いられてきた領域でのオブジェクト指向プログラミングを実現する言語である.

本研究はRuby言語を使用しています.大きな理由としては

\begin{itemize}
\tightlist
\item
  構文の自由度が高く,記述量が少なくて済む.
\item
  強力な標準ライブラリが備えられている.
\end{itemize}

Rubyは変数の型付けがないため,記述量を少なく済ませることができ,"gem"という形式で公開されているライブラリが豊富かつ強力なので本研究はRuby言語を使用しました.

    \section{RubyGems}\label{rubygems}

Rubygemの基本的な説明は以下の通り,RubyGemsは,Ruby言語用のパッケージ管理システムであり,Rubyのプログラムと("gem"と呼ばれる)ライブラリの配布用標準フォーマットを提供している.gemを容易に管理でき,gemを配布するサーバの機能も持つ.

本研究ではRubyGemsのgemを利用してファイル操作やパスの受け取りなどを行い,本研究で開発したソフトもgemに公開してある.

    \section{Keybind}\label{keybind}

Keybindの基本的な説明は以下の通り,押下するキー(単独キーまたは複数キーの組み合わせ)と,実行される機能との対応関係のことである.また,キーを押下したときに実行させる機能を割り当てる行為のことである.

以下controlを押しながらをc-と記述する.本研究におけるKeybindの習熟はCUI操作の習熟に酷似している.カーソル移動においてもGUIベースでマウスを使い行の先頭をクリックするより,CUIによりc-aを押すことで即座に行の先頭にカーソルを持っていくことができる.習熟するのであれば,どちらの方が早いかは一目瞭然である.本研究はKeybindの習熟によるCUI操作の適応で作業の効率化,高速化に重点を置いている.

    \section{CUI(Character User
Interface)}\label{cuicharacter-user-interface}

CUIは,キーボード等からの文字列を入力とし,文字列が表示されるウィンドウや古くはラインプリンタで印字される文字などを出力とする,ユーザインタフェースの様式で,GUI(Graphical
User Interface)の対義語として使われる.

CUIとGUIにはそれぞれ大きな違いがある.GUIの利点は以下の通り,

\begin{itemize}
\tightlist
\item
  文字だけでなくアイコンなどの絵も表示できる.
\item
  対象物が明確な点や,マウスで比較的簡単に操作できる.
\item
  即座に操作結果が反映される.
\end{itemize}

CUIの利点は以下の通り,

\begin{itemize}
\tightlist
\item
  コマンドを憶えていれば複雑な処理が簡単に行える.
\item
  キーボードから手を離すことなく作業の高速化が行える.
\end{itemize}

今回GUIではなくCUI操作の習熟を目的にした理由は,

\begin{itemize}
\tightlist
\item
  コマンドを憶えることで作業効率が上がる.
\item
  editor操作の習熟も孕んでいるから.
\end{itemize}

カーソル移動においてもGUIではなくCUI操作により,ワンコマンドで動かした方が効率的である.上記の理由から,GUIではなくCUI操作の習熟を目的としている.

    \section{使用したgemファイル}\label{ux4f7fux7528ux3057ux305fgemux30d5ux30a1ux30a4ux30eb}

    \subsection{diff-lcs}\label{diff-lcs}

    diff-lcsは,二つのファイルの差分を求めて出力してくれる.テキストの差分を取得するメソッドは,Diff::LCS.sdiff
と Diff::LCS.diff
の2つがある.複数行の文字列を比較した場合の2つのメソッドの違いは以下のとおり.

\begin{enumerate}
\def\labelenumi{\arabic{enumi}.}
\tightlist
\item
  Diff::LCS.sdiff

  \begin{enumerate}
  \def\labelenumii{\arabic{enumii}.}
  \tightlist
  \item
    比較結果を1文字ずつ表示する
  \end{enumerate}
\item
  Diff::LCS.diff

  \begin{enumerate}
  \def\labelenumii{\arabic{enumii}.}
  \tightlist
  \item
    比較した結果,違いがあった行について,違いがあった箇所のみ表示する.
  \end{enumerate}
\end{enumerate}

今回使用したのは後者(Diff:LCS.diff)である.理由は間違った部分だけを表示した方が見やすいと考えたからである.

    \subsection{Thor}\label{thor}

    Thorは,コマンドラインツールの作成を支援するライブラリです.gitやbundlerのようなサブコマンドツールを簡単に作成することができます.
Thorの使用でサブコマンドを自然言語に近い形で覚えることができる.

    \subsection{Minitest}\label{minitest}

    Minitestはテストを自動化するためのテスト用のフレームワークである.Rubyにはいくつかのテスティングフレームワークがありますが,Minitestというフレームワークを利用した理由は以下の通りです.

\begin{enumerate}
\def\labelenumi{\arabic{enumi}.}
\tightlist
\item
  Rubyをインストールすると一緒にインストールされるため,特別なセットアップが不要.
\item
  学習コストが比較的低い.
\item
  Railsのデフォルトのテスティングフレームワークなので,Railsを開発するときにも知識を活かしやすい.
\end{enumerate}

上記の理由から,sequential\_checkではminitestを採用しております.

    \subsection{FileUtils}\label{fileutils}

    再帰的な削除などの基本的なファイル操作を行うためのライブラリ

    \subsection{open3}\label{open3}

    プログラムを実行し,そのプロセスの標準出力,標準入力,標準エラー出力にパイプをつなぐためのものである.

    \subsection{Bundler}\label{bundler}

Bundlerはアプリケーション谷で依存するgemパッケージを管理するためのツールです.1つのシステム上で複数のアプリケーションを開発する場合や,デプロイ時にアプリケーションに紐付けてgemパッケージを管理したい場合に利用される.

    \subsection{Rubocop}\label{rubocop}

RubocopはRubyのソースコード解析ツールである.Rubyスタイルガイドや他のスタイルガイドに準拠しているかどうかを自動チェックしてくれるソフトウェアです.自分が打ち込んだ問題文となるソースコードのチェックに使用しました.
