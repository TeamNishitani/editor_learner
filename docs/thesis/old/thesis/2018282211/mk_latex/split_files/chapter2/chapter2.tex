\chapter{基本的事項}\label{ux57faux672cux7684ux4e8bux9805}

    \section{Emacs}\label{emacs}

    本研究において使用するeditorはEmacsである.ツールはプログラマ自身の手の延長である.これは他のどのようなソフトウェアツールよりもEditorに対して当てはまる.テキストはプログラミングにおける最も基本的な生素材なので,できる限り簡単に操作できる必要があります.{[}1{]}そこで西谷研究室で勧められているEmacsの機能については以下の通りである,

\begin{enumerate}
\def\labelenumi{\arabic{enumi}.}
\tightlist
\item
  設定可能である. フォント,色,ウィンドウサイズ,キーバインドを含めた全ての外見が好みに応じて設定できるようになっていること.通常の操作がキーストロークだけで行えると,手をキーボードから離す必要がなくなり,結果的にマウスやメニュー駆動型のコマンドよりも効率的に操作できるようになります
\item
  拡張性がある. 新しいプログラミング言語が出てきただけで,使い物にならなくなるようなエディタではなく,どんな新しい言語やテキスト形式が出てきたとしても,その言語の意味合いを「教え込む」ことが可能です
\item
  プログラム可能であること. 込み入った複数の手順を実行できるよう,Editorはプログラム可能であることが必須である.
\end{enumerate}

これらの機能は本来エディタが持つべき基本的な機能である.これらに加えてEmacsは,

\begin{enumerate}
\def\labelenumi{\arabic{enumi}.}
\tightlist
\item
  構文のハイライト Rubyの構文にハイライトを入れたい場合はファイル名の後に.rbと入れることでRubyモードに切り替わり構文にハイライトを入れることが可能になる.
\item
  自動インデント. テキストを編集する際,改行時に自動的にスペースやタブなどを入力しインデント調整を行ってくれる.
\end{enumerate}

などのプログラミング言語に特化した特徴を備えています.強力なeditorを習熟することは生産性を高めることに他ならない.カーソルの移動にしても,1回のキー入力で単語単位,行単位,ブロック単位,関数単位でカーソルを移動させることができれば,一文字ずつ,あるいは一行ずつ繰り返してキー入力を行う場合とは効率が大きく変わってきます.Emacsはこれらの全ての機能を孕んでいてeditorとして非常に優秀である.よって本研究はEmacsをベースとして研究を進める.

    \section{Ruby}\label{ruby}

Rubyの基本的な説明は以下の通り,

\begin{verbatim}
Rubyはまつもとゆきひろにより開発されたオブジェクト指向スクリプト言語であり,スクリプト言語が用いられてきた領域でのオブジェクト指向プログラミングを実現する言語である.[1]
\end{verbatim}

本研究はRuby言語を使用しています.大きな理由としては *
構文の自由度が高く,記述量が少なくて済む. *
強力な標準ライブラリが備えられている.

Rubyは変数の型付けがないため,記述量を少なく済ませることができ,"gem"という形式で公開されているライブラリが豊富かつ強力なので本研究はRuby言語を使用しました.

    \section{RubyGems}\label{rubygems}

Rubygemの基本的な説明は以下の通り,

\begin{verbatim}
RubyGemsは,Ruby言語用のパッケージ管理システムであり,Rubyのプログラムと("gem"と呼ばれる)ライブラリの配布用標準フォーマットを提供している.gemを容易に管理でき,gemを配布するサーバの機能も持つ.[2]
\end{verbatim}

本研究ではRubyGemsのgemを利用してファイル操作やパスの受け取りなどを行う.

    \section{Keybind}\label{keybind}

Keybindの基本的な説明は以下の通り,

\begin{verbatim}
押下するキー(単独キーまたは複数キーの組み合わせ)と,実行される機能との対応関係のことである.また,キーを押下したときに実行させる機能を割り当てる行為のことである.[3]
\end{verbatim}

以下controlを押しながらをc-と記述する.本研究におけるKeybindの習熟はCUI操作の習熟に酷似している.カーソル移動においてもGUIベースでマウスを使い行の先頭をクリックするより,CUIによりc-aを押すことで即座に行の先頭にカーソルを持っていくことができる.習熟するのであれば,どちらの方が早いかは一目瞭然である.本研究はKeybindの習熟によるCUI操作の適応にも重点を置いている.

    \section{CUI(Character User
Interface)}\label{cuicharacter-user-interface}

CUIは,

\begin{verbatim}
キーボード等からの文字列を入力とし,文字列が表示されるウィンドウや古くはラインプリンタで印字される文字などを出力とする,ユーザインタフェースの様式で,GUI(Graphical User Interface)の対義語として使われる.[4]
\end{verbatim}

CUIとGUIにはそれぞれ大きな違いがある.GUIの利点は以下の通り,

\begin{itemize}
\tightlist
\item
  文字だけでなくアイコンなどの絵も表示できる.
\item
  対象物が明確な点や,マウスで比較的簡単に操作できる.
\item
  即座に操作結果が反映される.
\end{itemize}

CUIの利点は以下の通り,

\begin{itemize}
\tightlist
\item
  コマンドを憶えていれば複雑な処理が簡単に行える.
\item
  キーボードから手を離すことなく作業の高速化が行える.
\end{itemize}

今回GUIではなくCUI操作の習熟を目的にした理由は,

\begin{itemize}
\tightlist
\item
  コマンドを憶えることで作業効率が上がる.
\item
  editor操作の習熟も孕んでいるから.
\end{itemize}

カーソル移動においてもGUIではなくCUI操作により,ワンコマンドで動かした方が効率的である.上記の理由から,GUIではなくCUI操作の習熟を目的としている.

    \section{使用したgemファイル}\label{ux4f7fux7528ux3057ux305fgemux30d5ux30a1ux30a4ux30eb}

    \subsection{diff-lcs}\label{diff-lcs}

    diff-lcsは,二つのファイルの差分を求めて出力してくれる.テキストの差分を取得するメソッドは,Diff::LCS.sdiff
と Diff::LCS.diff
の2つがある.複数行の文字列を比較した場合の2つのメソッドの違いは以下のとおり.

\begin{enumerate}
\def\labelenumi{\arabic{enumi}.}
\tightlist
\item
  Diff::LCS.sdiff 比較結果を1文字ずつ表示する
\item
  Diff::LCS.diff
  比較した結果,違いがあった行について,違いがあった箇所のみ表示する.
\end{enumerate}

今回使用したのは後者(Diff:LCS.diff)である.理由は間違った部分だけを表示した方が見やすいと考えたからである.

    \subsection{Thor}\label{thor}

    Thorは,コマンドラインツールの作成を支援するライブラリです.gitやbundlerのようなサブコマンドツールを簡単に作成することができます.Thorの使用で容易にサブコマンドを直感的に覚えやすくした.

    \subsection{Minitest}\label{minitest}

    Minitestはテストを自動化するためのテスト用のフレームワークである.Rubyにはいくつかのテスティングフレームワークがありますが,Minitestというフレームワークを利用した理由は以下の通りです.

\begin{enumerate}
\def\labelenumi{\arabic{enumi}.}
\tightlist
\item
  Rubyをインストールすると一緒にインストールされるため,特別なセットアップが不要.
\item
  学習コストが比較的低い.
\item
  Railsのデフォルトのテスティングフレームワークなので,Railsを開発するときにも知識を活かしやすい.
\end{enumerate}

    \subsection{FileUtils}\label{fileutils}

    再帰的な削除などの基本的なファイル操作を行うためのライブラリ

    \subsection{open3}\label{open3}

    プログラムを実行し,そのプロセスの標準出力,標準入力,標準エラー出力にパイプをつなぐためのものである.

    \section{Bundler}\label{bundler}

Bundlerはアプリケーション谷で依存するgemパッケージを管理するためのツールです.1つのシステム上で複数のアプリケーションを開発する場合や,デプロイ時にアプリケーションに紐付けてgemパッケージを管理したい場合に利用される.

    \subsection{Rubocop}\label{rubocop}

RubocopはRubyのソースコード解析ツールである.Rubyスタイルガイドや他のスタイルガイドに準拠しているかどうかを自動チェックしてくれるソフトウェアです.自分が打ち込んだ問題文となるソースコードのチェックに使用しました.{[}{]}

    \section{editor\_learnerの概要}\label{editor_learnerux306eux6982ux8981}

    \section{Installation}\label{installation}

\subsection{githubによるinstall}\label{githubux306bux3088ux308binstall}

githubによるインストール方法は以下の通りである. 1.
"https://github.com/souki1103/editor\_learner" へアクセス 1. Clone or
downloadを押下,SSHのURLをコピー 1. コマンドラインにてgit clone
{[}コピーしたURL{]}を行う

上記の手順で開発したファイルがそのまま自分のディレクトリにインストールされる.

\subsection{gemによるinstall}\label{gemux306bux3088ux308binstall}

gemによるインストール方法は以下の通りである. 1. コマンドラインにてgem
install editor\_learnerと入力,実行 1.
ファイルがホームディレクトの.rbenv/versions/2.4.0/lib/ruby/gems/2.4.0/gemsにeditor\_learnerが収納される

これでeditor\_learnerとコマンドラインで入力することで実行可能となる.

    \section{uninstall}\label{uninstall}

\subsection{githubからinstallした場合のuninstall方法}\label{githubux304bux3089installux3057ux305fux5834ux5408ux306euninstallux65b9ux6cd5}

gituhubからinstallした場合のuninstall方法は以下の通りである.

\begin{enumerate}
\def\labelenumi{\arabic{enumi}.}
\tightlist
\item
  ホームディレクトで

  \begin{enumerate}
  \def\labelenumii{\arabic{enumii}.}
  \setcounter{enumii}{1}
  \tightlist
  \item
    rm -rf editor\_learnerを入力
  \end{enumerate}
\item
  ホームディレクトリからeditor\_learnerが削除されていることを確認する.
\end{enumerate}

以上がuninstall方法である.

\subsection{gemからinstallした場合のuninstall方法}\label{gemux304bux3089installux3057ux305fux5834ux5408ux306euninstallux65b9ux6cd5}

gemからinstallした場合のuninstall方法は以下の通りである.

\begin{enumerate}
\def\labelenumi{\arabic{enumi}.}
\tightlist
\item
  ターミナル上のコマンドラインで

  \begin{enumerate}
  \def\labelenumii{\arabic{enumii}.}
  \setcounter{enumii}{1}
  \tightlist
  \item
    gem uninstall editor\_learnerを入力
  \end{enumerate}
\item
  ホームディレクトの.rbenv/versions/2.4.0/lib/ruby/gems/2.4.0/gemsにeditor\_learnerが削除されていることを確認する.
\end{enumerate}

以上がuninstall方法である.

    \section{動作環境}\label{ux52d5ux4f5cux74b0ux5883}

    Rubyのversionが2.4.0以上でなければ動かない.理由としては,gemに格納されているパスを正しいく受け渡しできないからである.2.4.0以下で動作させるためにはeditor\_learnerの最新versionのみを入れることによって動作することが確認できている.

\subsection{error時の対処法}\label{errorux6642ux306eux5bfeux51e6ux6cd5}

errorが出た場合は以下の方法を試してください

\begin{enumerate}
\def\labelenumi{\arabic{enumi}.}
\tightlist
\item
  rm -rf editor\_learnerをコマンドラインで入力
\end{enumerate}

これによりファイル生成によるバグを解消できる.もう一つの方法は

\begin{enumerate}
\def\labelenumi{\arabic{enumi}.}
\tightlist
\item
  gem uninstall editor\_learnerをコマンドラインで入力
\item
  全てのversionをuninstallする.
\item
  再度gem install editor\_learnerで最新versionのみをinstallする.
\end{enumerate}

上記の手順によりRubyのversionによるバグが解消されることが確認できている.現在起こるであろうと予想されるバグの解消法は上記の2つである.Rubyのversionが2.4.0以上であればなんの不具合もなく動作することが確認できている.

    \section{初期設定}\label{ux521dux671fux8a2dux5b9a}

    特別な初期設定はほとんどないが起動方法は以下の通りである,

\begin{enumerate}
\def\labelenumi{\arabic{enumi}.}
\item
  コマンドライン上にてeditor\_learnerを入力する.

  \begin{enumerate}
  \def\labelenumii{\arabic{enumii}.}
  \setcounter{enumii}{1}
  \tightlist
  \item
    editor\_learnerを起動することでホームディレクトリにeditor\_learner/workshopと呼ばれるファイルが作成される.workshopは作業場という意味である.
  \item
    workshopの中にquestion.rbとanswer.rb,random\_h.rbとruby\_1\textsubscript{ruby\_6が作成され,ruby\_1}ruby\_6の中に1.rb\textasciitilde{}3.rbが作成されていることを確認する.
    {[}画像添付{]}
  \end{enumerate}
\item
  起動すると以下のようなサブコマンドの書かれた画面が表示されることを確認する.

  Commands: editor\_lerner delete {[}number\textasciitilde{}number{]}
  editor\_learner help {[}COMMAND{]} editor\_learner random\_check
  editor\_leraner sequential\_check {[}lesson\_number{]}
  {[}1\textasciitilde{}3numbers{]}
\item
  editor\_learnerの後にサブコマンドと必要に応じた引数を入力すると動作する.それぞれのサブコマンドの更に詳しい説明は以下の通りである.
\end{enumerate}

    \section{delete}\label{delete}

    editor\_learnerを起動することで初期設定で述べたようにホームディレクトリにeditor\_learner/workshopが作成される.deleteはworkshopに作成されたruby\_1\textasciitilde{}ruby\_6を削除するために作成されたものである.sequential\_checkで1度プログラムを作成してしまうと再度実行するとIt
have been
finished!と表示されてしまうので,削除するコマンドを作成しました.コマンド例は以下の通りである.

コマンド例

\begin{enumerate}
\def\labelenumi{\arabic{enumi}.}
\tightlist
\item
  editor\_learner delete 1 3
\end{enumerate}

上記のように入力することで1〜3までのファイルが削除される.サブコマンドの後の引数は2つの数字(char型)であり,削除するファイルの範囲を入力する.

    \section{random\_h.rbとsequential\_h.rb}\label{random_h.rbux3068sequential_h.rb}

random\_h.rbとsequential\_h.rbが初期設定で作成され,editor\_learnerを起動することで自動的に作成され,random\_checkとsequential\_checkを行う際に最初に開くファイルとなる.random\_check用とsequential\_check用に二つのファイルがある.random\_check用のファイルは以下の通りである.

random\_h.rb

\begin{verbatim}
# to open  question.rb
#  c-x 2: split window vertically
#  c-x c-f: find file and input question.rb
# open answer.rb as above
#  c-x 3: split window horizontally
#  c-x c-f: find file and input answer.rb
# move the other window
#  c-x o: other windw
# then edit answer.rb as question.rb

# c-a: move ahead
# c-d: delete character
# c-x c-s: save file
# c-x c-c: quit edit
# c-k: kill the line
# c-y: paste of killed line
\end{verbatim}

上から順に説明すると, 1. question.rbを開くためにc-x
2で画面を2分割にする. 1. c-x c-fでquestion.rbを探して開く. 1.
次にanswer.rbを開くために画面を3分割する 1. 同様にc-x
c-fでanswer.rbを探して開く. 1. c-x
oでanswer.rbを編集するためにポインタを移動させる. 1.
question.rbに書かれているコードをanswer.rbに写す.

これらの手順がrandom\_h.rbに記述されている.

次にsequential\_h.rb

\begin{verbatim}
#to open q.rb
# c-x 2: find file and input q.rb
# c-x c-f: find file and input q.rb
# open 1~3.rb as above
# c-x 3: split find file and input 1~3.rb
# move the other window
# c-x o: other window
# then edit 1~3.rb q.rb

# c-a:move ahead
# c-d: delete character
# c-x c-s: save file
# c-x c-c: quit edit
# c-k: kill the line
# c-y: paste of killed line
\end{verbatim}

書かれている内容自体はrandom\_h.rbとほとんど差異がないが,開くファイルの名前が違うため別のファイルとして作成された.この手順に沿って作業することになる.下に書かれているのは主要キーバインドであり,必要に応じて見て,使用する形となっている.

    \section{random\_checkの動作}\label{random_checkux306eux52d5ux4f5c}

random\_checkの動作開始から終了は以下の通りである.

\begin{enumerate}
\def\labelenumi{\arabic{enumi}.}
\tightlist
\item
  コマンドライン上にてeditor\_learne random\_checkを入力
\item
  新しいターミナル(ホームディレクトリ/editor\_learner/workshopから始まる)が開かれる.
\item
  random\_h.rbを開いてrandom\_h.rbに沿ってquestion.rbに書かれているコードをanswer.rbに写す.
\item
  前のターミナルに戻り,コマンドラインに"check"と入力することで正誤判定を行ってくれる.
\item
  間違っていればdiff-lcsにより間違った箇所が表示される.
\item
  正しければ新しいターミナルが開かれてから終了までの時間とIt have been
  finished!が表示され終了となる.
\end{enumerate}

更に次回random\_check起動時には前に書いたコードがanswer.rbに格納されたままなので全て削除するのではなく,前のコードの必要な部分は残すことができる.

random\_checkの大きな目的はtyping速度,正確性の向上,editor操作やRuby言語の習熟に重点を置いている.いかに早く終わらせるかのポイントがtyping速度,正確性とeditor操作である.

    \section{sequential\_checkの動作}\label{sequential_checkux306eux52d5ux4f5c}

    sequential\_checkの動作開始から終了は以下の通りである.

\begin{enumerate}
\def\labelenumi{\arabic{enumi}.}
\tightlist
\item
  コマンドライン上でeditor\_learner sequential\_check
  {[}1\textasciitilde{}6の数字{]} {[}1\textasciitilde{}3の数字{]}を入力
\item
  新しいターミナル(ホームディレクトリ/editor\_learner/workshop/ruby\_{[}1\textasciitilde{}6の数字{]})が開かれる.
\item
  sequential\_h.rbを開いてsequential\_h.rbに沿ってq.rbに書かれている内容を第2引数の数字.rbに写す.
\item
  前のターミナルに戻り,コマンドラインに"check"と入力することで正誤判定を行う.
\item
  間違っていれば間違った箇所が表示される.再度q.rbと第2引数の数字.rbを開いて間違った箇所を修正する.
\item
  正しければruby\_1/1.rb is done!のように表示される.
\end{enumerate}

sequential\_checkは1\textasciitilde{}3の順に1.rbがリファクタリングや追加され2.rbになり,完成形が3.rbになるといった形式である.連続的なプログラムの完成までを写経するのでsequential\_checkと名付けられた.

sequential\_checkの大きな目的はリファクタリングによるRuby言語の学習とCUI操作によるキーバインドの習熟,タイピング速度,正確性の向上に重点を置いている.コードがリファクタリングされる様を写経することで自分自身でRubyのコードを書くときに他の人が見やすくなるようなコードが書けるようになる.

    \section{実装コードの解説}\label{ux5b9fux88c5ux30b3ux30fcux30c9ux306eux89e3ux8aac}

    \section{起動時のプログラム}\label{ux8d77ux52d5ux6642ux306eux30d7ux30edux30b0ux30e9ux30e0}

editor\_learnerを起動したときに自動に動く部分である.コードは以下の通りである.

\begin{verbatim}
def initialize(*args)
      super
      @prac_dir="#{ENV['HOME']}/editor_learner/workshop"
      @lib_location = Open3.capture3("gem environment gemdir")
      @versions = Open3.capture3("gem list editor_learner")
      p @latest_version = @versions[0].chomp.gsub(' (', '-').gsub(')','')
      @inject = File.join(@lib_location[0].chomp, "/gems/#{@latest_version}/lib")
      if File.exist?(@prac_dir) != true then
        FileUtils.mkdir_p(@prac_dir)
        FileUtils.touch("#{@prac_dir}/question.rb")
        FileUtils.touch("#{@prac_dir}/answer.rb")
        FileUtils.touch("#{@prac_dir}/random_h.rb")
        if File.exist?("#{@inject}/random_h.rb") == true then
          FileUtils.cp("#{@inject}/random_h.rb", "#{@prac_dir}/random_h.rb")
        elsif
          FileUtils.cp("#{ENV['HOME']}/editor_learner/lib/random_h.rb", "#{@prac_dir}/random_h.rb")
        end
      end
      range = 1..6
      range_ruby = 1..3
      range.each do|num|
        if File.exist?("#{@prac_dir}/ruby_#{num}") != true then
          FileUtils.mkdir("#{@prac_dir}/ruby_#{num}")
          FileUtils.touch("#{@prac_dir}/ruby_#{num}/q.rb")
          FileUtils.touch("#{@prac_dir}/ruby_#{num}/sequential_h.rb")
          if File.exist?("#{@inject}/sequential_h.rb") == true then
            FileUtils.cp("#{@inject}/sequential_h.rb", "#{@prac_dir}/ruby_#{num}/sequential_h.rb")
          else
            FileUtils.cp("#{ENV['HOME']}/editor_learner/lib/sequential_h.rb", "#{@prac_dir}/ruby_#{num}/sequential_h.rb")
          end
          range_ruby.each do|n|
            FileUtils.touch("#{@prac_dir}/ruby_#{num}/#{n}.rb")
          end
        end
      end
    end
\end{verbatim}

この部分は基本的にディレクトリやファイルの作成が主である.上から順に説明すると,@prac\_dirはホームディレクトリ/editor\_learner/workshopを指しており,ファイルを作る際のパスとして作成されたインスタンス定数である.その後の3つのインスタンス定数(@lib\_location,@versions,@latest\_version)はgemでinstallされた場合ファイルの場所がホームディレクトリ/.rbenv/versions/2.4.0/lib/ruby/gems/2.4.0/gemsのeditor\_learnerに格納されているためgemでinstallした人とgithubでinstallした人とではパスが変わってしまうためこれらの3つのインスタンス定数を用意した.実際の振る舞いとしては,File.existによりprac\_dirがなければディレクトリを作成しさらにその中にquestion.rbとanswer.rbを作成する.gemにリリースしていることからgemでinstallした人とgithubでinstallした人のパスの違いをif文で条件分岐させている.これによりrandom\_h.rbを正常にコピーすることができた.

    \section{delete}\label{delete}

\begin{verbatim}
desc 'delete [number~number]', 'delete the ruby_file choose number to delet\
e file'

def delete(n, m)
  range = n..m
  range.each{|num|
  if File.exist?("#{@prac_dir}/ruby_#{num}") == true then
    system "rm -rf #{@prac_dir}/ruby_#{num}"
  end
  }
end
\end{verbatim}

    \section{random\_check}\label{random_check}

\begin{verbatim}
desc 'random_check', 'ramdom check your typing and edit skill.'
    def random_check(*argv)
      random = rand(1..15)
      p random
      s = "#{random}.rb"
      puts "check starting ..."
      puts "type following commands on the terminal"
      puts "> emacs question.rb answer.rb"

      src_dir = File.expand_path('../..', __FILE__) # "Users/souki/editor_learner"
      if File.exist?("#{@inject}/random_check_question/#{s}") == true then
        FileUtils.cp("#{@inject}/random_check_question/#{s}", "#{@prac_dir}/question.rb")
      elsif
        FileUtils.cp(File.join(src_dir, "lib/random_check_question/#{s}"),  "#{@prac_dir}/question.rb")
      end
      open_terminal
      
      start_time = Time.now
      loop do
        a = STDIN.gets.chomp
        if a == "check" && FileUtils.compare_file("#{@prac_dir}/question.rb", "#{@prac_dir}/answer.rb") == true then
          puts "It have been finished!"
          break
        elsif FileUtils.compare_file("#{@prac_dir}/question.rb", "#{@prac_dir}/answer.rb") != true then
          @inputdata = File.open("#{@prac_dir}/answer.rb").readlines
          @checkdata = File.open("#{@prac_dir}/question.rb").readlines
          diffs = Diff::LCS.diff("#{@inputdata}", "#{@checkdata}")
          diffs.each do |diff|
            p diff
          end
        end
      end
      end_time = Time.now
      time = end_time - start_time - 1
      
      puts "#{time} sec"
    end
\end{verbatim}

    \section{sequential\_check}\label{sequential_check}

\begin{verbatim}
desc 'sequential_check [lesson_number] [1~3number] ','sequential check your typing skill and edit skill choose number'
    def sequential_check(*argv, n, m)
      l = m.to_i - 1
     
      @seq_dir = "lib/sequential_check_question"
      q_rb = "ruby_#{n}/#{m}.rb"
      @seqnm_dir = File.join(@seq_dir,q_rb)
      @pracnm_dir = "#{ENV['HOME']}/editor_learner/workshop/ruby_#{n}/#{m}.rb"
      @seqnq_dir = "lib/sequential_check_question/ruby_#{n}/q.rb"
      @pracnq_dir = "#{ENV['HOME']}/editor_learner/workshop/ruby_#{n}/q.rb"      
      @seqnl_dir = "lib/sequential_check_question/ruby_#{n}/#{l}.rb"
      @pracnl_dir = "#{ENV['HOME']}/editor_learner/workshop/ruby_#{n}/#{l}.rb"      
      puts "check starting ..."
      puts "type following commands on the terminal"
      src_dir = File.expand_path('../..', __FILE__)
      if File.exist?("#{@inject}/sequential_check_question/ruby_#{n}/#{m}.rb") == true then
        FileUtils.cp("#{@inject}/sequential_check_question/ruby_#{n}/#{m}.rb", "#{@pracnq_dir}")
      elsif
        FileUtils.cp(File.join(src_dir, "#{@seqnm_dir}"),  "#{@pracnq_dir}")
      end
      if l != 0 && FileUtils.compare_file("#{@pracnm_dir}", "#{@pracnq_dir}") != true
        FileUtils.compare_file("#{@pracnl_dir}", (File.join(src_dir, "#{@seqnl_dir}"))) == true
        FileUtils.cp("#{@pracnl_dir}", "#{@pracnm_dir}")
      end
      
      if FileUtils.compare_file(@pracnm_dir, @pracnq_dir) != true then
        system "osascript -e 'tell application \"Terminal\" to do script \"cd #{@prac_dir}/ruby_#{n} \" '"
        loop do
          a = STDIN.gets.chomp
          if a == "check" && FileUtils.compare_file("#{@pracnm_dir}", "#{@pracnq_dir}") == true then
            puts "ruby_#{n}/#{m}.rb is done!"
            break
          elsif FileUtils.compare_file("#{@pracnm_dir}", "#{@pracnq_dir}") != true then
            @inputdata = File.open("#{@pracnm_dir}").readlines
            @checkdata = File.open("#{@pracnq_dir}").readlines
            diffs = Diff::LCS.diff("#{@inputdata}", "#{@checkdata}")
            diffs.each do |diff|
              p diff
            end
          end
        end
       else
        p "ruby_#{n}/#{m}.rb is finished!"
      end
    end
\end{verbatim}

    \section{open\_terminal}\label{open_terminal}

新しいターミナルを開くメソッドである.コードは以下の通りである.

\begin{verbatim}
def open_terminal
        pwd = Dir.pwd
        system "osascript -e 'tell application \"Terminal\" to do script \"cd #{@prac_dir} \" '"
      end
\end{verbatim}

新しく開かれたターミナルはprac\_dir(editor\_learner/workshop)のディレクトリからスタートするように設定されている.

    \section{他のソフトとの比較}\label{ux4ed6ux306eux30bdux30d5ux30c8ux3068ux306eux6bd4ux8f03}

    \section{総括}\label{ux7dcfux62ec}

    同じようなことを書く

    \section{謝辞}\label{ux8b1dux8f9e}

本研究を行うにあたり,終始多大なるご指導,御鞭撻をいただいた西谷滋人教授に対し,深く御礼申し上げます.また,本研究の進行に伴い,様々な助力,知識の供給をいただきました西谷研究室の同輩,先輩方に心から感謝の意を示します.本当にありがとうございました.

    \section{付録A
プログラムのソースコード}\label{ux4ed8ux9332a-ux30d7ux30edux30b0ux30e9ux30e0ux306eux30bdux30fcux30b9ux30b3ux30fcux30c9}

\begin{Shaded}
\begin{Highlighting}[]
\CommentTok{# coding: utf-8}
\NormalTok{require }\StringTok{'fileutils'}
\NormalTok{require }\StringTok{'colorize'}
\NormalTok{require }\StringTok{'thor'}
\NormalTok{require }\StringTok{"editor_learner/version"}
\NormalTok{require }\StringTok{'diff-lcs'}
\NormalTok{require }\StringTok{"open3"}


\KeywordTok{module} \DataTypeTok{EditorLearner}
\KeywordTok{class} \DataTypeTok{CLI}\NormalTok{ < }\DataTypeTok{Thor}

    \KeywordTok{def}\NormalTok{ initialize(*args)}
      \DecValTok{super}
      \OtherTok{@prac_dir}\NormalTok{=}\StringTok{"}\OtherTok{#\{}\DataTypeTok{ENV}\NormalTok{[}\StringTok{'HOME'}\NormalTok{]}\OtherTok{\}}\StringTok{/editor_learner/workshop"}
      \OtherTok{@lib_location}\NormalTok{ = }\DataTypeTok{Open3}\NormalTok{.capture3(}\StringTok{"gem environment gemdir"}\NormalTok{)}
      \OtherTok{@versions}\NormalTok{ = }\DataTypeTok{Open3}\NormalTok{.capture3(}\StringTok{"gem list editor_learner"}\NormalTok{)}
\NormalTok{      p }\OtherTok{@latest_version}\NormalTok{ = }\OtherTok{@versions}\NormalTok{[}\DecValTok{0}\NormalTok{].chomp.gsub(}\StringTok{' ('}\NormalTok{, }\CharTok{'-'}\NormalTok{).gsub(}\CharTok{')'}\NormalTok{,}\StringTok{''}\NormalTok{)}
      \OtherTok{@inject}\NormalTok{ = }\DataTypeTok{File}\NormalTok{.join(}\OtherTok{@lib_location}\NormalTok{[}\DecValTok{0}\NormalTok{].chomp, }\StringTok{"/gems/}\OtherTok{#\{@latest_version\}}\StringTok{/lib"}\NormalTok{)}
      \KeywordTok{if} \DataTypeTok{File}\NormalTok{.exist?(}\OtherTok{@prac_dir}\NormalTok{) != }\DecValTok{true} \KeywordTok{then}
        \DataTypeTok{FileUtils}\NormalTok{.mkdir_p(}\OtherTok{@prac_dir}\NormalTok{)}
        \DataTypeTok{FileUtils}\NormalTok{.touch(}\StringTok{"}\OtherTok{#\{@prac_dir\}}\StringTok{/question.rb"}\NormalTok{)}
        \DataTypeTok{FileUtils}\NormalTok{.touch(}\StringTok{"}\OtherTok{#\{@prac_dir\}}\StringTok{/answer.rb"}\NormalTok{)}
        \DataTypeTok{FileUtils}\NormalTok{.touch(}\StringTok{"}\OtherTok{#\{@prac_dir\}}\StringTok{/random_h.rb"}\NormalTok{)}
        \KeywordTok{if} \DataTypeTok{File}\NormalTok{.exist?(}\StringTok{"}\OtherTok{#\{@inject\}}\StringTok{/random_h.rb"}\NormalTok{) == }\DecValTok{true} \KeywordTok{then}
          \DataTypeTok{FileUtils}\NormalTok{.cp(}\StringTok{"}\OtherTok{#\{@inject\}}\StringTok{/random_h.rb"}\NormalTok{, }\StringTok{"}\OtherTok{#\{@prac_dir\}}\StringTok{/random_h.rb"}\NormalTok{)}
        \KeywordTok{elsif}  
          \DataTypeTok{FileUtils}\NormalTok{.cp(}\StringTok{"}\OtherTok{#\{}\DataTypeTok{ENV}\NormalTok{[}\StringTok{'HOME'}\NormalTok{]}\OtherTok{\}}\StringTok{/editor_learner/lib/random_h.rb"}\NormalTok{, }\StringTok{"}\OtherTok{#\{@prac_dir\}}\StringTok{/random_h.rb"}\NormalTok{)}
        \KeywordTok{end}
      \KeywordTok{end}
\NormalTok{      range = }\DecValTok{1}\NormalTok{..}\DecValTok{6}
\NormalTok{      range_ruby = }\DecValTok{1}\NormalTok{..}\DecValTok{3}
\NormalTok{      range.each }\KeywordTok{do}\NormalTok{|num|}
        \KeywordTok{if} \DataTypeTok{File}\NormalTok{.exist?(}\StringTok{"}\OtherTok{#\{@prac_dir\}}\StringTok{/ruby_}\OtherTok{#\{}\NormalTok{num}\OtherTok{\}}\StringTok{"}\NormalTok{) != }\DecValTok{true} \KeywordTok{then}
          \DataTypeTok{FileUtils}\NormalTok{.mkdir(}\StringTok{"}\OtherTok{#\{@prac_dir\}}\StringTok{/ruby_}\OtherTok{#\{}\NormalTok{num}\OtherTok{\}}\StringTok{"}\NormalTok{)}
          \DataTypeTok{FileUtils}\NormalTok{.touch(}\StringTok{"}\OtherTok{#\{@prac_dir\}}\StringTok{/ruby_}\OtherTok{#\{}\NormalTok{num}\OtherTok{\}}\StringTok{/q.rb"}\NormalTok{)}
          \DataTypeTok{FileUtils}\NormalTok{.touch(}\StringTok{"}\OtherTok{#\{@prac_dir\}}\StringTok{/ruby_}\OtherTok{#\{}\NormalTok{num}\OtherTok{\}}\StringTok{/sequential_h.rb"}\NormalTok{)}
          \KeywordTok{if} \DataTypeTok{File}\NormalTok{.exist?(}\StringTok{"}\OtherTok{#\{@inject\}}\StringTok{/sequential_h.rb"}\NormalTok{) == }\DecValTok{true} \KeywordTok{then}
            \DataTypeTok{FileUtils}\NormalTok{.cp(}\StringTok{"}\OtherTok{#\{@inject\}}\StringTok{/sequential_h.rb"}\NormalTok{, }\StringTok{"}\OtherTok{#\{@prac_dir\}}\StringTok{/ruby_}\OtherTok{#\{}\NormalTok{num}\OtherTok{\}}\StringTok{/sequential_h.rb"}\NormalTok{)}
          \KeywordTok{else}
            \DataTypeTok{FileUtils}\NormalTok{.cp(}\StringTok{"}\OtherTok{#\{}\DataTypeTok{ENV}\NormalTok{[}\StringTok{'HOME'}\NormalTok{]}\OtherTok{\}}\StringTok{/editor_learner/lib/sequential_h.rb"}\NormalTok{, }\StringTok{"}\OtherTok{#\{@prac_dir\}}\StringTok{/ruby_}\OtherTok{#\{}\NormalTok{num}\OtherTok{\}}\StringTok{/sequential_h.rb"}\NormalTok{)}
          \KeywordTok{end}
\NormalTok{          range_ruby.each }\KeywordTok{do}\NormalTok{|n|}
            \DataTypeTok{FileUtils}\NormalTok{.touch(}\StringTok{"}\OtherTok{#\{@prac_dir\}}\StringTok{/ruby_}\OtherTok{#\{}\NormalTok{num}\OtherTok{\}}\StringTok{/}\OtherTok{#\{}\NormalTok{n}\OtherTok{\}}\StringTok{.rb"}\NormalTok{)}
          \KeywordTok{end}
        \KeywordTok{end}
      \KeywordTok{end}
    \KeywordTok{end}
    
\NormalTok{    desc }\StringTok{'delete [number~number]'}\NormalTok{, }\StringTok{'delete the ruby_file choose number to delete file'}
    
    \KeywordTok{def}\NormalTok{ delete(n, m)}
\NormalTok{      range = n..m}
\NormalTok{      range.each\{|num|}
      \KeywordTok{if} \DataTypeTok{File}\NormalTok{.exist?(}\StringTok{"}\OtherTok{#\{@prac_dir\}}\StringTok{/ruby_}\OtherTok{#\{}\NormalTok{num}\OtherTok{\}}\StringTok{"}\NormalTok{) == }\DecValTok{true} \KeywordTok{then}
\NormalTok{        system }\StringTok{"rm -rf }\OtherTok{#\{@prac_dir\}}\StringTok{/ruby_}\OtherTok{#\{}\NormalTok{num}\OtherTok{\}}\StringTok{"}
      \KeywordTok{end}
\NormalTok{      \}}
    \KeywordTok{end}

\NormalTok{    desc }\StringTok{'sequential_check [lesson_number] [1~3number] '}\NormalTok{,}\StringTok{'sequential check your typing skill and edit skill choose number'}
    \KeywordTok{def}\NormalTok{ sequential_check(*argv, n, m)}
\NormalTok{      l = m.to_i - }\DecValTok{1}
     
      \OtherTok{@seq_dir}\NormalTok{ = }\StringTok{"lib/sequential_check_question"}
\NormalTok{      q_rb = }\StringTok{"ruby_}\OtherTok{#\{}\NormalTok{n}\OtherTok{\}}\StringTok{/}\OtherTok{#\{}\NormalTok{m}\OtherTok{\}}\StringTok{.rb"}
      \OtherTok{@seqnm_dir}\NormalTok{ = }\DataTypeTok{File}\NormalTok{.join(}\OtherTok{@seq_dir}\NormalTok{,q_rb)}
      \OtherTok{@pracnm_dir}\NormalTok{ = }\StringTok{"}\OtherTok{#\{}\DataTypeTok{ENV}\NormalTok{[}\StringTok{'HOME'}\NormalTok{]}\OtherTok{\}}\StringTok{/editor_learner/workshop/ruby_}\OtherTok{#\{}\NormalTok{n}\OtherTok{\}}\StringTok{/}\OtherTok{#\{}\NormalTok{m}\OtherTok{\}}\StringTok{.rb"}
      \OtherTok{@seqnq_dir}\NormalTok{ = }\StringTok{"lib/sequential_check_question/ruby_}\OtherTok{#\{}\NormalTok{n}\OtherTok{\}}\StringTok{/q.rb"}
      \OtherTok{@pracnq_dir}\NormalTok{ = }\StringTok{"}\OtherTok{#\{}\DataTypeTok{ENV}\NormalTok{[}\StringTok{'HOME'}\NormalTok{]}\OtherTok{\}}\StringTok{/editor_learner/workshop/ruby_}\OtherTok{#\{}\NormalTok{n}\OtherTok{\}}\StringTok{/q.rb"}      
      \OtherTok{@seqnl_dir}\NormalTok{ = }\StringTok{"lib/sequential_check_question/ruby_}\OtherTok{#\{}\NormalTok{n}\OtherTok{\}}\StringTok{/}\OtherTok{#\{}\NormalTok{l}\OtherTok{\}}\StringTok{.rb"}
      \OtherTok{@pracnl_dir}\NormalTok{ = }\StringTok{"}\OtherTok{#\{}\DataTypeTok{ENV}\NormalTok{[}\StringTok{'HOME'}\NormalTok{]}\OtherTok{\}}\StringTok{/editor_learner/workshop/ruby_}\OtherTok{#\{}\NormalTok{n}\OtherTok{\}}\StringTok{/}\OtherTok{#\{}\NormalTok{l}\OtherTok{\}}\StringTok{.rb"}      
\NormalTok{      puts }\StringTok{"check starting ..."}
\NormalTok{      puts }\StringTok{"type following commands on the terminal"}
\NormalTok{      src_dir = }\DataTypeTok{File}\NormalTok{.expand_path(}\StringTok{'../..'}\NormalTok{, }\DecValTok{__FILE__}\NormalTok{)}
      \KeywordTok{if} \DataTypeTok{File}\NormalTok{.exist?(}\StringTok{"}\OtherTok{#\{@inject\}}\StringTok{/sequential_check_question/ruby_}\OtherTok{#\{}\NormalTok{n}\OtherTok{\}}\StringTok{/}\OtherTok{#\{}\NormalTok{m}\OtherTok{\}}\StringTok{.rb"}\NormalTok{) == }\DecValTok{true} \KeywordTok{then}
        \DataTypeTok{FileUtils}\NormalTok{.cp(}\StringTok{"}\OtherTok{#\{@inject\}}\StringTok{/sequential_check_question/ruby_}\OtherTok{#\{}\NormalTok{n}\OtherTok{\}}\StringTok{/}\OtherTok{#\{}\NormalTok{m}\OtherTok{\}}\StringTok{.rb"}\NormalTok{, }\StringTok{"}\OtherTok{#\{@pracnq_dir\}}\StringTok{"}\NormalTok{)}
      \KeywordTok{elsif}
        \DataTypeTok{FileUtils}\NormalTok{.cp(}\DataTypeTok{File}\NormalTok{.join(src_dir, }\StringTok{"}\OtherTok{#\{@seqnm_dir\}}\StringTok{"}\NormalTok{),  }\StringTok{"}\OtherTok{#\{@pracnq_dir\}}\StringTok{"}\NormalTok{)}
      \KeywordTok{end}
      \KeywordTok{if}\NormalTok{ l != }\DecValTok{0}\NormalTok{ && }\DataTypeTok{FileUtils}\NormalTok{.compare_file(}\StringTok{"}\OtherTok{#\{@pracnm_dir\}}\StringTok{"}\NormalTok{, }\StringTok{"}\OtherTok{#\{@pracnq_dir\}}\StringTok{"}\NormalTok{) != }\DecValTok{true}
        \DataTypeTok{FileUtils}\NormalTok{.compare_file(}\StringTok{"}\OtherTok{#\{@pracnl_dir\}}\StringTok{"}\NormalTok{, (}\DataTypeTok{File}\NormalTok{.join(src_dir, }\StringTok{"}\OtherTok{#\{@seqnl_dir\}}\StringTok{"}\NormalTok{))) == }\DecValTok{true}
        \DataTypeTok{FileUtils}\NormalTok{.cp(}\StringTok{"}\OtherTok{#\{@pracnl_dir\}}\StringTok{"}\NormalTok{, }\StringTok{"}\OtherTok{#\{@pracnm_dir\}}\StringTok{"}\NormalTok{)}
      \KeywordTok{end}
      
      \KeywordTok{if} \DataTypeTok{FileUtils}\NormalTok{.compare_file(}\OtherTok{@pracnm_dir}\NormalTok{, }\OtherTok{@pracnq_dir}\NormalTok{) != }\DecValTok{true} \KeywordTok{then}
\NormalTok{        system }\StringTok{"osascript -e 'tell application \textbackslash{}"Terminal\textbackslash{}" to do script \textbackslash{}"cd }\OtherTok{#\{@prac_dir\}}\StringTok{/ruby_}\OtherTok{#\{}\NormalTok{n}\OtherTok{\}}\StringTok{ \textbackslash{}" '"}
\NormalTok{        loop }\KeywordTok{do}
\NormalTok{          a = }\DataTypeTok{STDIN}\NormalTok{.gets.chomp}
          \KeywordTok{if}\NormalTok{ a == }\StringTok{"check"}\NormalTok{ && }\DataTypeTok{FileUtils}\NormalTok{.compare_file(}\StringTok{"}\OtherTok{#\{@pracnm_dir\}}\StringTok{"}\NormalTok{, }\StringTok{"}\OtherTok{#\{@pracnq_dir\}}\StringTok{"}\NormalTok{) == }\DecValTok{true} \KeywordTok{then}
\NormalTok{            puts }\StringTok{"ruby_}\OtherTok{#\{}\NormalTok{n}\OtherTok{\}}\StringTok{/}\OtherTok{#\{}\NormalTok{m}\OtherTok{\}}\StringTok{.rb is done!"}
            \KeywordTok{break}
          \KeywordTok{elsif} \DataTypeTok{FileUtils}\NormalTok{.compare_file(}\StringTok{"}\OtherTok{#\{@pracnm_dir\}}\StringTok{"}\NormalTok{, }\StringTok{"}\OtherTok{#\{@pracnq_dir\}}\StringTok{"}\NormalTok{) != }\DecValTok{true} \KeywordTok{then}
            \OtherTok{@inputdata}\NormalTok{ = }\DataTypeTok{File}\NormalTok{.open(}\StringTok{"}\OtherTok{#\{@pracnm_dir\}}\StringTok{"}\NormalTok{).readlines}
            \OtherTok{@checkdata}\NormalTok{ = }\DataTypeTok{File}\NormalTok{.open(}\StringTok{"}\OtherTok{#\{@pracnq_dir\}}\StringTok{"}\NormalTok{).readlines}
\NormalTok{            diffs = }\DataTypeTok{Diff}\NormalTok{::}\DataTypeTok{LCS}\NormalTok{.diff(}\StringTok{"}\OtherTok{#\{@inputdata\}}\StringTok{"}\NormalTok{, }\StringTok{"}\OtherTok{#\{@checkdata\}}\StringTok{"}\NormalTok{)}
\NormalTok{            diffs.each }\KeywordTok{do}\NormalTok{ |diff|}
\NormalTok{              p diff}
            \KeywordTok{end}
          \KeywordTok{end}
        \KeywordTok{end}
       \KeywordTok{else}
\NormalTok{        p }\StringTok{"ruby_}\OtherTok{#\{}\NormalTok{n}\OtherTok{\}}\StringTok{/}\OtherTok{#\{}\NormalTok{m}\OtherTok{\}}\StringTok{.rb is finished!"}
      \KeywordTok{end}
    \KeywordTok{end}

\NormalTok{    desc }\StringTok{'random_check'}\NormalTok{, }\StringTok{'ramdom check your typing and edit skill.'}
    \KeywordTok{def}\NormalTok{ random_check(*argv)}
\NormalTok{      random = rand(}\DecValTok{1}\NormalTok{..}\DecValTok{15}\NormalTok{)}
\NormalTok{      p random}
\NormalTok{      s = }\StringTok{"}\OtherTok{#\{}\NormalTok{random}\OtherTok{\}}\StringTok{.rb"}
\NormalTok{      puts }\StringTok{"check starting ..."}
\NormalTok{      puts }\StringTok{"type following commands on the terminal"}
\NormalTok{      puts }\StringTok{"> emacs question.rb answer.rb"}

\NormalTok{      src_dir = }\DataTypeTok{File}\NormalTok{.expand_path(}\StringTok{'../..'}\NormalTok{, }\DecValTok{__FILE__}\NormalTok{) }\CommentTok{# "Users/souki/editor_learner"}
      \KeywordTok{if} \DataTypeTok{File}\NormalTok{.exist?(}\StringTok{"}\OtherTok{#\{@inject\}}\StringTok{/random_check_question/}\OtherTok{#\{}\NormalTok{s}\OtherTok{\}}\StringTok{"}\NormalTok{) == }\DecValTok{true} \KeywordTok{then}
        \DataTypeTok{FileUtils}\NormalTok{.cp(}\StringTok{"}\OtherTok{#\{@inject\}}\StringTok{/random_check_question/}\OtherTok{#\{}\NormalTok{s}\OtherTok{\}}\StringTok{"}\NormalTok{, }\StringTok{"}\OtherTok{#\{@prac_dir\}}\StringTok{/question.rb"}\NormalTok{)}
      \KeywordTok{else}
        \DataTypeTok{FileUtils}\NormalTok{.cp(}\DataTypeTok{File}\NormalTok{.join(src_dir, }\StringTok{"lib/random_check_question/}\OtherTok{#\{}\NormalTok{s}\OtherTok{\}}\StringTok{"}\NormalTok{),  }\StringTok{"}\OtherTok{#\{@prac_dir\}}\StringTok{/question.rb"}\NormalTok{)}
      \KeywordTok{end}
\NormalTok{      open_terminal}
      
\NormalTok{      start_time = }\DataTypeTok{Time}\NormalTok{.now}
\NormalTok{      loop }\KeywordTok{do}
\NormalTok{        a = }\DataTypeTok{STDIN}\NormalTok{.gets.chomp}
        \KeywordTok{if}\NormalTok{ a == }\StringTok{"check"}\NormalTok{ && }\DataTypeTok{FileUtils}\NormalTok{.compare_file(}\StringTok{"}\OtherTok{#\{@prac_dir\}}\StringTok{/question.rb"}\NormalTok{, }\StringTok{"}\OtherTok{#\{@prac_dir\}}\StringTok{/answer.rb"}\NormalTok{) == }\DecValTok{true} \KeywordTok{then}
\NormalTok{          puts }\StringTok{"It have been finished!"}
          \KeywordTok{break}
        \KeywordTok{elsif} \DataTypeTok{FileUtils}\NormalTok{.compare_file(}\StringTok{"}\OtherTok{#\{@prac_dir\}}\StringTok{/question.rb"}\NormalTok{, }\StringTok{"}\OtherTok{#\{@prac_dir\}}\StringTok{/answer.rb"}\NormalTok{) != }\DecValTok{true} \KeywordTok{then}
          \OtherTok{@inputdata}\NormalTok{ = }\DataTypeTok{File}\NormalTok{.open(}\StringTok{"}\OtherTok{#\{@prac_dir\}}\StringTok{/answer.rb"}\NormalTok{).readlines}
          \OtherTok{@checkdata}\NormalTok{ = }\DataTypeTok{File}\NormalTok{.open(}\StringTok{"}\OtherTok{#\{@prac_dir\}}\StringTok{/question.rb"}\NormalTok{).readlines}
\NormalTok{          diffs = }\DataTypeTok{Diff}\NormalTok{::}\DataTypeTok{LCS}\NormalTok{.diff(}\StringTok{"}\OtherTok{#\{@inputdata\}}\StringTok{"}\NormalTok{, }\StringTok{"}\OtherTok{#\{@checkdata\}}\StringTok{"}\NormalTok{)}
\NormalTok{          diffs.each }\KeywordTok{do}\NormalTok{ |diff|}
\NormalTok{            p diff}
          \KeywordTok{end}
        \KeywordTok{end}
      \KeywordTok{end}
\NormalTok{      end_time = }\DataTypeTok{Time}\NormalTok{.now}
\NormalTok{      time = end_time - start_time - }\DecValTok{1}
      
\NormalTok{      puts }\StringTok{"}\OtherTok{#\{}\NormalTok{time}\OtherTok{\}}\StringTok{ sec"}
    \KeywordTok{end}
    
\NormalTok{    no_commands }\KeywordTok{do}
      \KeywordTok{def}\NormalTok{ open_terminal}
\NormalTok{        pwd = }\DataTypeTok{Dir}\NormalTok{.pwd}
\NormalTok{        system }\StringTok{"osascript -e 'tell application \textbackslash{}"Terminal\textbackslash{}" to do script \textbackslash{}"cd }\OtherTok{#\{@prac_dir\}}\StringTok{ \textbackslash{}" '"}
      \KeywordTok{end}
    \KeywordTok{end}
  \KeywordTok{end}
\KeywordTok{end}
\end{Highlighting}
\end{Shaded}

    \section{参考文献}\label{ux53c2ux8003ux6587ux732e}


    % Add a bibliography block to the postdoc
    
    
    
    \end{document}
