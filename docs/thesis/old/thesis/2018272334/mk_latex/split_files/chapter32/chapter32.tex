\chapter{実装コードの解説}\label{ux5b9fux88c5ux30b3ux30fcux30c9ux306eux89e3ux8aac}

    \section{起動時のプログラム}\label{ux8d77ux52d5ux6642ux306eux30d7ux30edux30b0ux30e9ux30e0}

editor\_learnerを起動したときに自動に動く部分である.コードは以下の通りである.

\begin{verbatim}
def initialize(*args)
      super
      @prac_dir="#{ENV['HOME']}/editor_learner/workshop"
      @lib_location = Open3.capture3("gem environment gemdir")
      @versions = Open3.capture3("gem list editor_learner")
      p @latest_version = @versions[0].chomp.gsub(' (', '-').gsub(')','')
      @inject = File.join(@lib_location[0].chomp, "/gems/#{@latest_version}/lib")
      if File.exist?(@prac_dir) != true then
        FileUtils.mkdir_p(@prac_dir)
        FileUtils.touch("#{@prac_dir}/question.rb")
        FileUtils.touch("#{@prac_dir}/answer.rb")
        FileUtils.touch("#{@prac_dir}/random_h.rb")
        if File.exist?("#{@inject}/random_h.rb") == true then
          FileUtils.cp("#{@inject}/random_h.rb", "#{@prac_dir}/random_h.rb")
        elsif
          FileUtils.cp("#{ENV['HOME']}/editor_learner/lib/random_h.rb", "#{@prac_dir}/random_h.rb")
        end
      end
      range = 1..6
      range_ruby = 1..3
      range.each do|num|
        if File.exist?("#{@prac_dir}/ruby_#{num}") != true then
          FileUtils.mkdir("#{@prac_dir}/ruby_#{num}")
          FileUtils.touch("#{@prac_dir}/ruby_#{num}/q.rb")
          FileUtils.touch("#{@prac_dir}/ruby_#{num}/sequential_h.rb")
          if File.exist?("#{@inject}/sequential_h.rb") == true then
            FileUtils.cp("#{@inject}/sequential_h.rb", "#{@prac_dir}/ruby_#{num}/sequential_h.rb")
          else
            FileUtils.cp("#{ENV['HOME']}/editor_learner/lib/sequential_h.rb", "#{@prac_dir}/ruby_#{num}/sequential_h.rb")
          end
          range_ruby.each do|n|
            FileUtils.touch("#{@prac_dir}/ruby_#{num}/#{n}.rb")
          end
        end
      end
    end
\end{verbatim}

この部分は基本的にディレクトリやファイルの作成が主である.上から順に説明すると,@prac\_dirはホームディレクトリ/editor\_learner/workshopを指しており,ファイルを作る際のパスとして作成されたインスタンス定数である.その後の3つのインスタンス定数(@lib\_location,@versions,@latest\_version)はgemでinstallされた場合ファイルの場所がホームディレクトリ/.rbenv/versions/2.4.0/lib/ruby/gems/2.4.0/gemsのeditor\_learnerに格納されているためgemでinstallした人とgithubでinstallした人とではパスが変わってしまうためこれらの3つのインスタンス定数を用意した.実際の振る舞いとしては,File.existによりprac\_dirがなければディレクトリを作成しさらにその中にquestion.rbとanswer.rbを作成する.gemにリリースしていることからgemでinstallした人とgithubでinstallした人のパスの違いをif文で条件分岐させている.これによりrandom\_h.rbを正常にコピーすることができた.

    \section{delete}\label{delete}

\begin{verbatim}
desc 'delete [number~number]', 'delete the ruby_file choose number to delet\
e file'

def delete(n, m)
  range = n..m
  range.each{|num|
  if File.exist?("#{@prac_dir}/ruby_#{num}") == true then
    system "rm -rf #{@prac_dir}/ruby_#{num}"
  end
  }
end
\end{verbatim}

    \section{random\_check}\label{random_check}

\begin{verbatim}
desc 'random_check', 'ramdom check your typing and edit skill.'
    def random_check(*argv)
      random = rand(1..15)
      p random
      s = "#{random}.rb"
      puts "check starting ..."
      puts "type following commands on the terminal"
      puts "> emacs question.rb answer.rb"

      src_dir = File.expand_path('../..', __FILE__) # "Users/souki/editor_learner"
      if File.exist?("#{@inject}/random_check_question/#{s}") == true then
        FileUtils.cp("#{@inject}/random_check_question/#{s}", "#{@prac_dir}/question.rb")
      elsif
        FileUtils.cp(File.join(src_dir, "lib/random_check_question/#{s}"),  "#{@prac_dir}/question.rb")
      end
      open_terminal
      
      start_time = Time.now
      loop do
        a = STDIN.gets.chomp
        if a == "check" && FileUtils.compare_file("#{@prac_dir}/question.rb", "#{@prac_dir}/answer.rb") == true then
          puts "It have been finished!"
          break
        elsif FileUtils.compare_file("#{@prac_dir}/question.rb", "#{@prac_dir}/answer.rb") != true then
          @inputdata = File.open("#{@prac_dir}/answer.rb").readlines
          @checkdata = File.open("#{@prac_dir}/question.rb").readlines
          diffs = Diff::LCS.diff("#{@inputdata}", "#{@checkdata}")
          diffs.each do |diff|
            p diff
          end
        end
      end
      end_time = Time.now
      time = end_time - start_time - 1
      
      puts "#{time} sec"
    end
\end{verbatim}

    \section{sequential\_check}\label{sequential_check}

\begin{verbatim}
desc 'sequential_check [lesson_number] [1~3number] ','sequential check your typing skill and edit skill choose number'
    def sequential_check(*argv, n, m)
      l = m.to_i - 1
     
      @seq_dir = "lib/sequential_check_question"
      q_rb = "ruby_#{n}/#{m}.rb"
      @seqnm_dir = File.join(@seq_dir,q_rb)
      @pracnm_dir = "#{ENV['HOME']}/editor_learner/workshop/ruby_#{n}/#{m}.rb"
      @seqnq_dir = "lib/sequential_check_question/ruby_#{n}/q.rb"
      @pracnq_dir = "#{ENV['HOME']}/editor_learner/workshop/ruby_#{n}/q.rb"      
      @seqnl_dir = "lib/sequential_check_question/ruby_#{n}/#{l}.rb"
      @pracnl_dir = "#{ENV['HOME']}/editor_learner/workshop/ruby_#{n}/#{l}.rb"      
      puts "check starting ..."
      puts "type following commands on the terminal"
      src_dir = File.expand_path('../..', __FILE__)
      if File.exist?("#{@inject}/sequential_check_question/ruby_#{n}/#{m}.rb") == true then
        FileUtils.cp("#{@inject}/sequential_check_question/ruby_#{n}/#{m}.rb", "#{@pracnq_dir}")
      elsif
        FileUtils.cp(File.join(src_dir, "#{@seqnm_dir}"),  "#{@pracnq_dir}")
      end
      if l != 0 && FileUtils.compare_file("#{@pracnm_dir}", "#{@pracnq_dir}") != true
        FileUtils.compare_file("#{@pracnl_dir}", (File.join(src_dir, "#{@seqnl_dir}"))) == true
        FileUtils.cp("#{@pracnl_dir}", "#{@pracnm_dir}")
      end
      
      if FileUtils.compare_file(@pracnm_dir, @pracnq_dir) != true then
        system "osascript -e 'tell application \"Terminal\" to do script \"cd #{@prac_dir}/ruby_#{n} \" '"
        loop do
          a = STDIN.gets.chomp
          if a == "check" && FileUtils.compare_file("#{@pracnm_dir}", "#{@pracnq_dir}") == true then
            puts "ruby_#{n}/#{m}.rb is done!"
            break
          elsif FileUtils.compare_file("#{@pracnm_dir}", "#{@pracnq_dir}") != true then
            @inputdata = File.open("#{@pracnm_dir}").readlines
            @checkdata = File.open("#{@pracnq_dir}").readlines
            diffs = Diff::LCS.diff("#{@inputdata}", "#{@checkdata}")
            diffs.each do |diff|
              p diff
            end
          end
        end
       else
        p "ruby_#{n}/#{m}.rb is finished!"
      end
    end
\end{verbatim}

    \section{open\_terminal}\label{open_terminal}

新しいターミナルを開くメソッドである.コードは以下の通りである.

\begin{verbatim}
def open_terminal
        pwd = Dir.pwd
        system "osascript -e 'tell application \"Terminal\" to do script \"cd #{@prac_dir} \" '"
      end
\end{verbatim}

新しく開かれたターミナルはprac\_dir(editor\_learner/workshop)のディレクトリからスタートするように設定されている.

    