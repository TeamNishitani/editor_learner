\chapter{実装コードの解説}\label{ux5b9fux88c5ux30b3ux30fcux30c9ux306eux89e3ux8aac}

本章では,今回作成したプログラムをライブラリ化し継続的な発展が可能なようにそれぞれの処理の解説を記述する.

    \section{起動時に動作するプログラム}\label{ux8d77ux52d5ux6642ux306bux6bceux56deux52d5ux4f5cux3059ux308bux30d7ux30edux30b0ux30e9ux30e0}

initializeと名前に付けられたメソッドは特殊なメソッドでクラス内に記述した場合にはオブジェクトが作成されるときに自動的の呼び出される.editor\_learnerを動作したとき自動的に呼び出される部分である.

\begin{screen}
{\small
\begin{verbatim}
def initialize(*args)
      super
      @prac_dir="#{ENV['HOME']}/editor_learner/workshop"
      @lib_location = Open3.capture3("gem environment gemdir")
      @versions = Open3.capture3("gem list editor_learner")
      p @latest_version = @versions[0].chomp.gsub(' (', '-').gsub(')','')
      @inject = File.join(@lib_location[0].chomp, "/gems/#{@latest_version}
      /lib")
      if File.exist?(@prac_dir) != true then
        FileUtils.mkdir_p(@prac_dir)
        FileUtils.touch("#{@prac_dir}/question.rb")
        FileUtils.touch("#{@prac_dir}/answer.rb")
        FileUtils.touch("#{@prac_dir}/random_h.rb")
        if File.exist?("#{@inject}/random_h.rb") == true then
          FileUtils.cp("#{@inject}/random_h.rb", "#{@prac_dir}/random_h.rb")
        elsif
          FileUtils.cp("#{ENV['HOME']}/editor_learner/lib/random_h.rb", 
          "#{@prac_dir}/random_h.rb")
        end
      end
      range = 1..6
      range_ruby = 1..3
      range.each do|num|
        if File.exist?("#{@prac_dir}/ruby_#{num}") != true then
          FileUtils.mkdir("#{@prac_dir}/ruby_#{num}")
          FileUtils.touch("#{@prac_dir}/ruby_#{num}/q.rb")
          FileUtils.touch("#{@prac_dir}/ruby_#{num}/sequential_h.rb")
          if File.exist?("#{@inject}/sequential_h.rb") == true then
            FileUtils.cp("#{@inject}/sequential_h.rb", "#{@prac_dir}
            /ruby_#{num}/sequential_h.rb")
          else
            FileUtils.cp("#{ENV['HOME']}/editor_learner/lib/sequential_h.rb",
             "#{@prac_dir}/ruby_#{num}/sequential_h.rb")
          end
          range_ruby.each do|n|
\end{verbatim}}
\end{screen}
\begin{screen}
{\small
\begin{verbatim}
            FileUtils.touch("#{@prac_dir}/ruby_#{num}/#{n}.rb")
          end
        end
      end
    end
\end{verbatim}}
\end{screen}

この部分は基本的にディレクトリやファイルの作成が主である.上から順に説明すると,@prac\_dirはホームディレクトリ/editor\_learner/workshopを指しており,ファイルを作る際のパスとして作成されたインスタンス定数である.その後の3つのインスタンス定数(@lib\_location,@versions,@latest\_version)はgemでinstallされた場合ファイルの場所がホームディレクトリ/.rbenv/versions/2.4.0/lib/ruby/gems/2.4.0/gemsのeditor\_learnerに格納されているためgemでinstallした人とgithubでinstallした人とではパスが変わってしまうためこれらの3つのインスタンス定数を利用し@injectを用意した.これにより,gemからinstallした場合でも正しく動作するようになった.実際の振る舞いとしては,File.existによりprac\_dirがなければディレクトリを作成しさらにその中にquestion.rbとanswer.rbを作成する.gemにリリースしていることからgemでinstallした人とgithubでinstallした人のパスの違いをif文で条件分岐させている.これによりrandom\_h.rbやsequential\_h.rbを正常にコピーすることができた.

\subsection{プログラム内のインスタンス変数の概要}\label{ux30d7ux30edux30b0ux30e9ux30e0ux5185ux306eux30a4ux30f3ux30b9ux30bfux30f3ux30b9ux5909ux6570ux306eux6982ux8981}

'@'で始まる変数はインスタンス変数であり,特定のオブジェクトに所属している.インスタンス変数はそのクラスまたはサブクラスのメソッドから参照できる.初期化されない孫スタンス変数を参照した時の値はnillである.

このメソッドで使用されているインスタンス変数は5つである.prac\_dirはホームディレクトリ/editor\_learner/workshopを指しており,必要なファイルをここに作るのでもっとも基本的なパスとして受け渡すインスタンス変数となっている.その後の4つのインスタンス変数はgemからinstallした場合における,editor\_learnerが格納されているパスを受け渡すためのインスタンス変数である.一つずつの説明は以下の通り,

\begin{description}
\def\labelenumi{\arabic{enumi}.}
\tightlist
\item[lib\_location]
ターミナル上で"gem environment gemdir"を入力した場合に出力されるパスを格納している.(自分のターミナル場で実行すると/Users/souki/.rbenv/versions/2.4.0/
  lib/ruby/gems/2.4.0)最初はsystem"gem environment gemdir"に代入していたがsystemの返す値は文字列ではなく'true"と'false"である.なので,ここでopen3が必要となった.
\item[versions]
 gemでinstallされたeditor\_learnerのversionを受け取るためのパスを格納したインスタンス変数である.Rubyのversionが2.4.0以上であるならば,正しく最新版のパスを入手できるが,それ以下だと正しくパスを入手できないのでRubyのversionが2.4.0以上必要となる.versionをあげたくない人にはeditor\_learnerのgemを最新版のみinstallしていれば正しくパスを受け取ることができる.
\item[latest\_versions]
 versionsで受け取ったeditor\_learnerのversionの最新部分のパスを格納したインスタンス変数である.
\item[inject]
実際にこれらのパスをつなぎ合わせてできるgemでinstallされたeditor\_learnerが格納されているパスが格納されているインスタンス変数である.(自分の場合は/Users/souki/.rbenv
  /versions/2.4.0/lib/ruby/gems/2.4.0/gems/editor\_learner-1.1.2となる)
\end{description}
injectはいろんなメソッドで使われるため,普通の変数では他のメソッドで呼び出せないためインスタンス変数で書かれている.

\subsection{Fileの作成}\label{fileux306eux4f5cux6210}

全てのパスの準備が整ったら実際に作業する場所に必要なファイル(question.rbやanswer.rbなど)の作成が行われる.本研究のinitializeメソッドではeditor\_learner/workshopがホームディレクトリになければ作成する.さらに,その中にrandom\_checkに必要なファイル(question.rb,answer.rb,random\_h.rb)が作成される.random\_h.rbはgemでinstallした場合はeditor\_learnerの格納されている部分からコピーを行なっている.次に,sequential\_checkに必要なファイルを作成する.editor\_learner/workshopにruby\_1〜ruby\_6がなければ作成し,その中に1.rb〜3.rbとq.rb(問題をコピーするためのファイル)とsequential\_h.rbが作成される.sequential\_h.rbはrandom\_h.rbと同じでgemからinstallした場合はeditor\_learnerの格納されている部分からコピーを行なっている.このメソッドの大きな役割はファイル作成であり,この部分がなければ全てのプログラムが動作しない.

    \section{ファイル削除処理delete}\label{ux30d5ux30a1ux30a4ux30ebux524aux9664ux51e6ux7406delete}

sequential\_checkで終了したchapterとその中の問題(1.rb〜3.rb)をもう一度行いたい場合に一度書き込んでいるので「(例)ruby\_1/1.rb is done!」と表示されてしまう.なので終わった問題を再度やりたい場合には一度ファイルを削除しなければいけないので,deleteメソッドを作成した.deleteメソッドの大きな役割はsequential\_checkで打ち込みが完了したファイルの削除である.ソースコードは以下の通り,

\begin{screen}
{\small
\begin{verbatim}
desc 'delete [number~number]', 'delete the ruby_file choose number to delet\
e file'

def delete(n, m)
  range = n..m
  range.each{|num|
  if File.exist?("#{@prac_dir}/ruby_#{num}") == true then
    system "rm -rf #{@prac_dir}/ruby_#{num}"
  end
  }
end
\end{verbatim}}
\end{screen}

コード自体はいたってシンプルで引数を2つ受け取ることでその間の範囲のFileを削除するようなコードとなっている.systemの"rm -rf ファイル名"がファイルを削除するコマンドとなっている.まず,そのファイルが存在しているかどうかの判定をFile.existで行う.そして,存在しているなら引数として受け取ったnとmが範囲となっておりその範囲でfor文の働きをするeachメソッドを使用することで削除を再帰的に行なっている.

    \section{random\_check}\label{random_check}

random\_checkのコードは以下の通り,

\begin{screen}
{\small
\begin{verbatim}
desc 'random_check', 'ramdom check your typing and edit skill.'
    def random_check(*argv)
      random = rand(1..15)
      p random
      s = "#{random}.rb"
      puts "check starting ..."
      puts "type following commands on the terminal"
      puts "> emacs question.rb answer.rb"

      src_dir = File.expand_path('../..', __FILE__) 
      # "Users/souki/editor_learner"
      if File.exist?("#{@inject}/random_check_question/#{s}") == true then
        FileUtils.cp("#{@inject}/random_check_question/#{s}", 
        "#{@prac_dir}/question.rb")
      elsif
        FileUtils.cp(File.join(src_dir, "lib/random_check_question/#{s}"),  
        "#{@prac_dir}/question.rb")
      end
      open_terminal
      
      start_time = Time.now
      loop do
        a = STDIN.gets.chomp
        if a == "check" && FileUtils.compare_file("#{@prac_dir}/question.rb", 
        "#{@prac_dir}/answer.rb") == true then
          puts "It have been finished!"
          break
        elsif FileUtils.compare_file("#{@prac_dir}/question.rb",
         "#{@prac_dir}/answer.rb") != true then
          @inputdata = File.open("#{@prac_dir}/answer.rb").readlines
          @checkdata = File.open("#{@prac_dir}/question.rb").readlines
          diffs = Diff::LCS.diff("#{@inputdata}", "#{@checkdata}")
          diffs.each do |diff|
            p diff
          end
        end
      end
\end{verbatim}}
\end{screen}
\begin{screen}
{\small
\begin{verbatim}
      end_time = Time.now
      time = end_time - start_time - 1
      
      puts "#{time} sec"
    end
\end{verbatim}}
\end{screen}

random\_checkの概要を簡単に説明すると15個あるRubyのコードから15の乱数を取得し,選ばれた数字のファイルが問題としてコピーされて,それをanswer.rbに入力することで正解していたら新しいターミナルが開かれてから終了までの時間を評価する仕組みとなっている.

上から解説を行うと,1〜15のrandomな乱数を取得,起動と同時に選ばれた数字のファイルを表示する.そして,src\_dirでホームディレクトリ/editor\_learnerのパスが代入される.File.expand\_path()は,
\begin{quotation}
第2引数のパスを起点に,第1引数のパスを,フルパスに展開するメソッドである.\_\_FILE\_\_にはそのファイルのパスが入っているが,これは絶対パスとは限らない,そこで,File.expand\_path()の第2引数にこれを指定し,第1引数にそのファイルからの相対パスを書くことで,目的のファイルの絶対パスが得られる\cite{expand}.
\end{quotation}
ここで相対パスを得ることでFile.joinで必要なパスと結合させる.そして,gemでinstallした人とgithubからcloneした場合によるファイルのパスの違いをifで条件分岐.そして,15個あるソースコードから乱数によって選ばれたファイルがquestion.rbにコピーされる.コピーされた後に新しいターミナルが開かれ,時間計測が開始する.そして,answer.rbにquestion.rbと同じ内容を書き込む.FileUtils.compare\_fileを使うことで tabなどによる空白などもきちんと判定する.そうしなければ,改行などをめちゃくちゃにしても正しいと判断されてしまうからである.見にくいプログラムを書くと,他の人が見たとき理解しずらいので改行などもきちんと判定している.全てうち終えることで,前のターミナルに戻り"check"と入力する.checkを前のターミナルに入力できるようにgetsを使った.初めにgetsだけを使用した時改行が入ってしまいうまく入力できなかった.しかし,chompを入れることで改行をなくすことに成功.しかし,argvとgetsを併用することが不可能なことが判明した.そこでgetsの前にSTDINを入れることでargvとの併用が可能なことがわかり,STDIN.gets.chompと入力することでargvとの併用が可能となり,キーボードからの入力を受け取ることができた.そして,checkが入力されてかつFileUtils.compare\_fileで比較の結果が正しければ時間計測を終了し"it have been finished!"を表示する.間違っていた場合はfile.openによりquestion.rbとanswer.rbをinputとoutputに格納されてDiff::LCSのdiffによって間違っている箇所だけを表示する.間違った部分を見て,再度コードを打ち直し正しいと判定されるとbreakで終了する.一連の流れは異常である.

    \section{sequential\_check}\label{sequential_check}

sequential\_checkの場合はリファクタリングにあたりたくさんのインスタンス定数を作った.コードは以下の通り,

\begin{screen}
{\small
\begin{verbatim}
desc 'sequential_check [lesson_number] [1~3number] ',
'sequential check your typing skill and edit skill choose number'
    def sequential_check(*argv, n, m)
      l = m.to_i - 1
     
      @seq_dir = "lib/sequential_check_question"
      q_rb = "ruby_#{n}/#{m}.rb"
      @seqnm_dir = File.join(@seq_dir,q_rb)
      @pracnm_dir = "#{ENV['HOME']}/editor_learner/workshop/ruby_#{n}/#{m}.rb"
      @seqnq_dir = "lib/sequential_check_question/ruby_#{n}/q.rb"
      @pracnq_dir = "#{ENV['HOME']}/editor_learner/workshop/ruby_#{n}/q.rb"      
      @seqnl_dir = "lib/sequential_check_question/ruby_#{n}/#{l}.rb"
      @pracnl_dir = "#{ENV['HOME']}/editor_learner/workshop/ruby_#{n}/#{l}.rb"      
      puts "check starting ..."
      puts "type following commands on the terminal"
      src_dir = File.expand_path('../..', __FILE__)
      if File.exist?("#{@inject}/sequential_check_question/ruby_#{n}/#{m}.rb")
       == true then
        FileUtils.cp("#{@inject}/sequential_check_question/ruby_#{n}/#{m}.rb"
        , "#{@pracnq_dir}")
      elsif
        FileUtils.cp(File.join(src_dir, "#{@seqnm_dir}"),  "#{@pracnq_dir}")
      end
      if l != 0 && FileUtils.compare_file("#{@pracnm_dir}", "#{@pracnq_dir}")
       != true
        FileUtils.compare_file("#{@pracnl_dir}", (File.join(src_dir,
         "#{@seqnl_dir}"))) == true
        FileUtils.cp("#{@pracnl_dir}", "#{@pracnm_dir}")
      end
      
      if FileUtils.compare_file(@pracnm_dir, @pracnq_dir) != true then
        system "osascript -e 'tell application \"Terminal\" to do script 
        \"cd #{@prac_dir}/ruby_#{n} \" '"
        loop do
          a = STDIN.gets.chomp
          if a == "check" && FileUtils.compare_file("#{@pracnm_dir}", 
          "#{@pracnq_dir}") == true then
            puts "ruby_#{n}/#{m}.rb is done!"
            break
\end{verbatim}}
\end{screen}
\begin{screen}
{\small
\begin{verbatim}            
          elsif FileUtils.compare_file("#{@pracnm_dir}", "#{@pracnq_dir}")
           != true then
            @inputdata = File.open("#{@pracnm_dir}").readlines
            @checkdata = File.open("#{@pracnq_dir}").readlines
            diffs = Diff::LCS.diff("#{@inputdata}", "#{@checkdata}")
            diffs.each do |diff|
              p diff
            end
          end
        end
       else
        p "ruby_#{n}/#{m}.rb is finished!"
      end
    end
\end{verbatim}}
\end{screen}

\subsection{インスタンス定数に格納されたパス}\label{ux30a4ux30f3ux30b9ux30bfux30f3ux30b9ux5b9aux6570ux306bux683cux7d0dux3055ux308cux305fux30d1ux30b9}

インスタンス定数に格納されているパスについての説明は上から順に以下の通り,

\begin{enumerate}
\def\labelenumi{\arabic{enumi}.}
\tightlist
\item
  seq\_dirはgithubでcloneした人が問題をコピーするときに使うパスである.
\item
  seqnm\_dirはその名の通りseq\_dirに引数であるnとmを代入したパスである.例として引数に1と1が代入された時は以下の通り,

  \begin{enumerate}
  \def\labelenumii{\arabic{enumii}.}
  \tightlist
  \item
    editor\_learner/sequential\_check\_question/ruby\_1/1.rbとなる.
  \end{enumerate}
\item
  pracnm\_dirはprac\_dirに二つの引数nとmを代入したものである.実際に作業するところのパスとして使用する.例として引数として1と1が代入された時は以下の通り,

  \begin{enumerate}
  \def\labelenumii{\arabic{enumii}.}
  \tightlist
  \item
    ホームディレクトリ/editor\_learner/workshop/ruby\_1/1.rbが格納される.
  \end{enumerate}
\item
  同様にseqとpracの後についている文字はその後のruby\_(数字)/(数字).rbの数字に入る文字を後につけている.
 \item
  例として,seqnl\_dirはruby\_n/l.rbのことである.以下,同じように引数の2つの数字がseqとpracの後についている.
\end{enumerate}

\subsection{動作部分}\label{ux52d5ux4f5cux90e8ux5206}

まずgemでinstallした場合とgithubでinstallした場合による違いを条件分岐によりパスを変えている.さらに1.rbが終了していた場合2.rbに1.rbをコピーした状態から始まるように処理が行われている.理由は前のコードがどのように変わったか,リファクタリングされたかがわかりやすいように前のコードをコピーしている.その後は"check"が入力された時かつFileUtils.compare\_fileで正しければ終了.間違っていればDiff::LCSで間違っている箇所を表示.もう一度修正し,"check"を入力,正解していれば終了."check"による正誤判定などはrandom\_checkと同様の書き方をしている.以上が一連のコードの解説である.

    \section{新しいターミナルを開くopen\_terminal}\label{ux65b0ux3057ux3044ux30bfux30fcux30dfux30caux30ebux3092ux958bux304fopen_terminal}

新しいターミナルを開くメソッドである.コードは以下の通りである.
\begin{screen}
{\small
\begin{verbatim}
def open_terminal
  pwd = Dir.pwd
  system "osascript -e 'tell application \"Terminal\" to do script \
  "cd #{@prac_dir} \" '"
end
\end{verbatim}}
\end{screen}

新しく開かれたターミナルはprac\_dir(editor\_learner/workshop)のディレクトリからスタートするように設定されている.random\_checkではeditor\_learner/workshopでターミナルが開かれ,sequential\_checkではeditor\_learner/workshop/第1引数で入力されたファイルの場所が開かれるようになっている.

    