\chapter{はじめに}\label{ux306fux3058ux3081ux306b}

    \section{研究の目的}\label{ux7814ux7a76ux306eux76eeux7684}

    editor\_learnerの開発の大きな目的はeditor(Emacs)操作,CUI操作(キーバインドなど),Ruby言語の習熟とタイピング速度,正確性の向上である.editor上で動かすためファイルの開閉,保存,画面分割といったCUI操作を習熟することができ,Ruby言語のプログラムを写経することでRuby言語の習熟へと繋げる.更にコードを打つことで正しい運指を身につけタイピング速度の向上も図っている.コードを打つ際にキーバインドを利用することでキーボードから手を離すことなくカーソル移動などのコマンドをGUIではなくCUI操作で行うことにより作業の効率化にも力を入れている.これら全てはプログラマにとって作業を効率化させるだけでなく,プログラマとしての質の向上につながる.

    \section{研究の動機}\label{ux7814ux7a76ux306eux52d5ux6a5f}

    初めはタッチタイピングを習得した経験を活かして,西谷によって開発されたshunkuntype(ターミナル上で実行するタイピングソフト)の再開発をテーマにしていたが,これ以上タイピングに特化したソフトを開発しても同じようなものがWeb上に大量に転がっており,そのようなものをいくつも開発しても意味がなく,それ以外の付加価値を付けたソフトを開発しようと考えた.西谷研究室ではタイピング,Ruby言語,Emacsによるeditor操作,CUI操作の習熟が作業効率に非常に大きな影響を与えるので習熟を勧めている.そこでこれらの習熟を目的としたソフトを開発しようと考えた.

    